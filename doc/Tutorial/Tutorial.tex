\documentclass[twoside]{book}

\usepackage[T1]{fontenc}
\usepackage[utf8]{inputenc}
\usepackage[english]{babel}

\usepackage{epigraph}
\setlength{\epigraphwidth}{0.5\textwidth}

\usepackage{color}
\usepackage{xcolor}
\usepackage{datetime}
\usepackage{listings}
\usepackage{fullpage}
\usepackage{framed,color}
\definecolor{shadecolor}{rgb}{0.9,0.9,0.9}

\lstset{language=haskell,frame=single}

\usepackage {hyperref}
\hypersetup{
   pdftitle={ViperVM - Tutorial},
   pdfauthor={Sylvain Henry},
   unicode=true,
   bookmarksnumbered=false,
   bookmarksopen=false,
   breaklinks=false,
   pdfborder={0 0 0},
   linktoc=all,
   colorlinks=true,
   citecolor=blue,
   linkcolor=blue,
   filecolor=blue,
   urlcolor=blue
}

\newcommand{\summary}[1]{%
   \begin{shaded}
      \noindent\textbf{\Large{}Summary}\\
      \indent#1
   \end{shaded}
}
\newcommand{\eg}{\textit{eg.~}}

\newcommand{\executeiffilenewer}[3]{%
\ifnum\pdfstrcmp{\pdffilemoddate{#1}}{\pdffilemoddate{#2}}>0{%
  \immediate\write18{#3}}%
\fi%
}

\newcommand{\iffilenewer}[3]{%
\ifnum\pdfstrcmp{\pdffilemoddate{#1}}{\pdffilemoddate{#2}}>0 %
  {#3}%
\fi%
}

\newcommand{\inkscape}[1]{%
  \immediate\write18{inkscape -z -D --file=#1.svg --export-pdf=#1.pdf}%
}

\newcommand{\inkscapeid}[2]{%
  \immediate\write18{inkscape -z -D -i #2 --file=#1.svg --export-pdf=#1_#2.pdf}%
}

\newcommand{\inkscapetex}[1]{%
  \immediate\write18{inkscape -z -D --file=#1.svg --export-pdf=#1.pdf --export-latex}%
}

\newcommand{\inkscapetexid}[2]{%
  \immediate\write18{%
    inkscape -z -D -i #2 --file=#1.svg --export-pdf=#1_#2.pdf --export-latex}%
}

\newcommand{\includesvg}[2][]{%
\IfFileExists{#2.pdf}%
  {\iffilenewer{#2.svg}{#2.pdf}{\inkscape{#2}}}%
  {\inkscape{#2}}%
\includegraphics[#1]{#2.pdf}%
}

\newcommand{\includesvgtex}[2][1.0]{%
\IfFileExists{#2.pdf}%
  {\iffilenewer{#2.svg}{#2.pdf}{\inkscapetex{#2}}}%
  {\inkscapetex{#2}}%
\def\svgwidth{#1\columnwidth}%
\input{#2.pdf_tex}%
}

\newcommand{\includesvgid}[3][]{%
\IfFileExists{#2_#3.pdf}%
  {\iffilenewer{#2.svg}{#2_#3.pdf}{\inkscapeid{#2}{#3}}}%
  {\inkscapeid{#2}{#3}}%
\includegraphics[#1]{#2_#3.pdf}%
}

\newcommand{\includesvgtexid}[3][1.0]{%
\IfFileExists{#2_#3.pdf}%
  {\iffilenewer{#2.svg}{#2_#3.pdf}{\inkscapetexid{#2}{#3}}}%
  {\inkscapetexid{#2}{#3}}%
\def\svgwidth{#1\columnwidth}%
\input{#2_#3.pdf_tex}%
}


\title{ViperVM - Tutorial}
\author{Sylvain Henry}
\date{\today{}\\\currenttime}

\begin{document}
\maketitle

\frontmatter

\tableofcontents{}

\mainmatter

\chapter{Presentation of the project}

The evolution of computer architectures let to the widespread of the
heterogeneous ones: those containing a multi-core CPU (often with non-uniform
memory access, NUMA) associated to a set of accelerators such as graphic
processing units (GPU), synergistic processing units (SPU), Xeon PHI, etc.
Programming these architectures is challenging because each unit has its own
features and the whole computer can be any combination of different units. In
addition, the way the units are interconnected is a factor that has to be taken
into account too.

ViperVM project aims at providing a environment to implement and test programming
models, scheduling algorithms, data management policies, etc. This document
presents the different steps that have led to the current design. It should be
evolving at the same pace than the software. 

I hope you will enjoy it! Any comment on this document is appreciated and should
be sent to me (Sylvain Henry, \url{hsyl20@gmail.com}).

\vfill
TODO
\begin{itemize}
   \item Cabal (init, configure, build, install)
   \item GHC options (-Wall -threaded)
   \item Licenses
\end{itemize}

\chapter{Architecture Model: Memory}

The first thing we need to decide is how the architecture will be represented
into the environment. 

In the OpenCL specification, an architecture is composed of several platforms,
themselves composed of several devices. Each device contains one memory and can
execute kernels. Obviously, it is flawed: platforms make no sense and are not
interoperable, device can contain several memories and processing units of
different kinds, etc. We will try to do a better job at defining an architecture
model.

The first entity we define is \texttt{Memory}. Perhaps a better would be
\texttt{Storage}. Each memory should be identifiable so we give them a unique
identifier. In addition, as there are many kinds of memory units
involved in architectures (hard disks, RAM, OpenCL device memory, etc.), we
introduce the algebraic data type \texttt{MemoryPeer} to store driver
specific details. We will add parameters to the constructor later for each
driver.

\begin{lstlisting}
type ID = Int

data Memory = Memory {
   memoryId :: ID,
   memoryPeer :: MemoryPeer
}

data MemoryPeer =
     HostMemory
   | CUDAMemory
   | OpenCLMemory
   | DiskMemory
\end{lstlisting}

Basically, the only thing a program can do with memory is to associate a region
of it with an opaque identifier: this is called allocating and the result is
called a buffer. Removing this association is called (buffer) releasing. Hence
we define a Buffer entity. Depending on the kind of memory we consider, the
low-level identifier will be different. For instance, for the host memory and
CUDA we get a pointer, for OpenCL we get an opaque clMem entity, for a hard disk
we get a file name\ldots These different identifiers are represented with the
algebraic data type \texttt{BufferPeer}.

\begin{lstlisting}
import Foreign.Ptr

data Buffer = Buffer {
   bufferMemory :: Memory,
   bufferPeer :: BufferPeer
}

data BufferPeer = 
     HostBuffer (Ptr ())
   | CUDABuffer
   | OpenCLBuffer
   | DiskBuffer
\end{lstlisting}

We would like to keep track of buffers that have been allocated in each memory.
To do that, we add a mutable field to the Memory type. We want to use software
transactional memory (STM) to manage future concurrent accesses to it, so we use
a \texttt{TVar} type. Don't forget to add the \texttt{stm} package to the list
of dependencies.

\begin{lstlisting}
import Control.Concurrent.STM

data Memory = Memory {
   ...
   memoryBuffers :: TVar [Buffer]
}
\end{lstlisting}

Now we can provide our first memory primitives: allocate and release. The code
for each driver is different, but for now we only code it for the host memory.
We use \texttt{undefined} to fill the blank and have a program that compiles
anyway. An allocation is only parameterized by the size of the desired memory
region and by the memory into which the allocation is performed. It returns
either the allocated buffer or an error. We decide to store the allocation size
in the buffer, hence the additional field. Finally, we set the type of a buffer
size to be \texttt{Word64} so that it should be large enough on most
architectures.

\begin{lstlisting}
import Data.Word

data Buffer = Buffer {
   ...
   bufferSize :: BufferSize
   ...
}

type BufferSize = Word64

data Result a b = Success a | Error b

data AllocError = 
     ErrAllocOutOfMemory
   | ErrAllocUnknown

allocate :: Word64 -> Memory -> IO (Result Buffer AllocError)
allocate size mem = do

   bufPeer <- case (memoryPeer mem) of
      HostMemory   -> allocateHost size mem
      CUDAMemory   -> undefined
      OpenCLMemory -> undefined
      DiskMemory   -> undefined

   case bufPeer of
      Error err -> return (Error err)
      Success peer -> do
         let buf = Buffer mem size peer
         atomically $ do
            bufs <- readTVar (memoryBuffers mem)
            writeTVar (memoryBuffers mem) (buf:bufs)
         return (Success buf)
\end{lstlisting}

Finally the "host" specific code to allocate a buffer is:

\begin{lstlisting}
{-# LANGUAGE ForeignFunctionInterface #-}

import Foreign.C.Types

foreign import ccall unsafe "stdlib.h malloc"  malloc :: CSize -> IO (Ptr a)
foreign import ccall unsafe "stdlib.h free"    free   :: Ptr a -> IO ()

-- | Allocate a buffer in host memory
allocateHost :: Word64 -> Memory -> IO (Result BufferPeer AllocError)
allocateHost size _ = do
   ptr <- malloc (fromIntegral size)
   return $ if ptr == nullPtr
      then Error ErrAllocOutOfMemory
      else Success (HostBuffer ptr)
\end{lstlisting}


To simplify the code a little bit, we introduce the \texttt{onSuccess} function
and we replace the second \texttt{case} in \texttt{allocate} with a call to it.

\begin{lstlisting}
import Control.Applicative ( (<$>) )

onSuccess :: Result a b -> (a -> IO c) -> IO (Result c b)
onSuccess (Success a) f = Success <$> f a
onSuccess (Error b) _ = return (Error b)

allocate size mem = do
...
   onSuccess bufPeer $ \peer -> do
      let buf = Buffer mem size peer
      atomically $ do
         bufs <- readTVar (memoryBuffers mem)
         writeTVar (memoryBuffers mem) (buf:bufs)
      return buf
\end{lstlisting}

Finally we can add the \texttt{release} primitive. Before we need to remove the
buffer from the list of allocated buffers of the memory it is allocated in.
Hence, we need to be able to find the buffer in the list and to delete it. To do
that, \texttt{Buffer} must be an instance of the \texttt{Eq} type class. We
first create an instance of \texttt{Eq} for \texttt{Memory} as it is easy given
there is a unique identifier field. Then we can automatically derive the
\texttt{Eq} instances for \texttt{BufferPeer} and \texttt{Buffer}.

\begin{lstlisting}
import Data.List

instance Eq Memory where
   (==) a b = memoryId a == memoryId b

data Buffer = Buffer {
   ...
} deriving (Eq)

data BufferPeer = ... deriving (Eq)

release :: Buffer -> IO ()
release buf = do
   atomically $ do
      let bufsVar = memoryBuffers (bufferMemory buf)
      bufs <- readTVar bufsVar
      writeTVar bufsVar (delete buf bufs)

   case bufferPeer buf of
      HostBuffer ptr -> free ptr
      CUDABuffer     -> undefined
      OpenCLBuffer   -> undefined
      DiskBuffer     -> undefined
\end{lstlisting}

Here is the full listing of our current \texttt{Platform} module.

\begin{lstlisting}
{-# LANGUAGE ForeignFunctionInterface #-}
module ViperVM.Platform.Platform (
   Memory(..), Buffer(..), Result(..),
   MemoryPeer(..), BufferPeer(..), AllocError(..),
   allocate, release
) where

import Control.Applicative ( (<$>) )
import Control.Concurrent.STM
import Data.List
import Data.Word
import Foreign.Ptr
import Foreign.C.Types

type ID = Int

data Memory = Memory {
   memoryId :: ID,
   memoryPeer :: MemoryPeer,
   memoryBuffers :: TVar [Buffer]
}

instance Eq Memory where
   (==) a b = memoryId a == memoryId b

data MemoryPeer =
     HostMemory
   | CUDAMemory
   | OpenCLMemory
   | DiskMemory

data Buffer = Buffer {
   bufferMemory :: Memory,
   bufferSize :: BufferSize,
   bufferPeer :: BufferPeer
} deriving (Eq)

data BufferPeer = 
     HostBuffer (Ptr ())
   | CUDABuffer
   | OpenCLBuffer
   | DiskBuffer
   deriving (Eq)

type BufferSize = Word64

data Result a b = Success a | Error b

onSuccess :: Result a b -> (a -> IO c) -> IO (Result c b)
onSuccess (Success a) f = Success <$> f a
onSuccess (Error b) _ = return (Error b)

data AllocError = 
     ErrAllocOutOfMemory
   | ErrAllocUnknown

-- | Allocate a buffer of the given size in the memory 
allocate :: Word64 -> Memory -> IO (Result Buffer AllocError)
allocate size mem = do

   bufPeer <- case (memoryPeer mem) of
      HostMemory   -> allocateHost size mem
      CUDAMemory   -> undefined
      OpenCLMemory -> undefined
      DiskMemory   -> undefined

   onSuccess bufPeer $ \peer -> do
      let buf = Buffer mem size peer
      atomically $ do
         bufs <- readTVar (memoryBuffers mem)
         writeTVar (memoryBuffers mem) (buf:bufs)
      return buf

foreign import ccall unsafe "stdlib.h malloc"  malloc :: CSize -> IO (Ptr a)
foreign import ccall unsafe "stdlib.h free"    free   :: Ptr a -> IO ()

-- | Allocate a buffer in host memory
allocateHost :: Word64 -> Memory -> IO (Result BufferPeer AllocError)
allocateHost size _ = do
   ptr <- malloc (fromIntegral size)
   return $ if ptr == nullPtr
      then Error ErrAllocOutOfMemory
      else Success (HostBuffer ptr)

-- | Release a buffer
release :: Buffer -> IO ()
release buf = do
   atomically $ do
      let bufsVar = memoryBuffers (bufferMemory buf)
      bufs <- readTVar bufsVar
      writeTVar bufsVar (delete buf bufs)

   case bufferPeer buf of
      HostBuffer ptr -> free ptr
      CUDABuffer     -> undefined
      OpenCLBuffer   -> undefined
      DiskBuffer     -> undefined
\end{lstlisting}

We can test that it is quite working with the following GHCI session:
\begin{lstlisting}
> import ViperVM.Platform.Platform
> import Control.Concurrent.STM
> import Control.Applicative
> m <- Memory 1 HostMemory <$> atomically (newTVar [])
> Success b <- allocate 10000 m
> length <$> atomically (readTVar (memoryBuffers m))
1
> Error err <- allocate 100000000000000 m
> length <$> atomically (readTVar (memoryBuffers m))
1
> release b
> length <$> atomically (readTVar (memoryBuffers m))
0
\end{lstlisting}

\chapter{Enters OpenCL}

\epigraph{
All different kinds with different ways\\
It would take a lifetime to explain\\
No one's exactly the same}
{No Doubt, \textit{Different People}}

In this chapter we will add support for OpenCL device memories. First we need to
understand a little bit how OpenCL platform model works. OpenCL basically
provides two API: \texttt{clGetPlatformIDs} to list available platforms and
\texttt{clGetDeviceIDs} to list available devices for each platform. In a
perfect world, for each device you could use \texttt{clCreateBuffer} and
\texttt{clReleaseMemObject} to allocate and release buffers (respectively). Well
in fact, it is not that easy: you have to create a \texttt{Context} entity (with
\texttt{clCreateContext} to create a group of devices before you can allocate a
buffer that is associated to the context and not to the device.

The first question that should come to mind is: how does the runtime system know
onto which device of the context it has to allocate the buffer? It is trivial if
there is only one device in the context, but otherwise it is impossible. So
buffers are lazily allocated on appropriate devices when another command (kernel
execution, etc.) requires them. As a consequence, out-of-capacity errors can be
returned by commands that do not perform any allocation. As it is hardly
manageable, we would like to force the allocation in the first place. Before
OpenCL 1.2, you cannot easily do it: you have to execute a command such as a
data transfer or a kernel execution involving the buffer and the device.
Starting with OpenCL 1.2, a new \texttt{clEnqueueMigrateMemObjects} method let
host code control where buffers have to be allocated. As for every OpenCL, this
one has to be submitted into a command queue (associated to a single device)
created with \texttt{clCreateCommandQueue}.

We could use the Foreign Function Interface (FFI) just like we did for
\texttt{malloc} previously. However, the difference is that OpenCL may not be
available on the target architecture. One solution is to add pragmas in our code
to decide at compilation time if OpenCL is enabled and compile the code
appropriately. This solution is not satisfactory because we could not distribute
a binary of our library. Instead we use dynamic linking so that programs using
our library could dynamically chose to use OpenCL or not.

Each hardware vendor can provide its own OpenCL implementation for its
accelerators in the form of a library exporting appropriate symbols. When
several accelerators from different vendors are available, the different
implementations have to be used in conjunction. To avoid mixing up symbols,
implementations of the OpenCL Installable Client Driver (ICD) provide a
multiplexer for OpenCL libraries. However it is quite badly designed, especially
when different implementations implement different versions of the OpenCL
specification.

Haskell provides dynamic linking methods for Posix compliant systems in the
\texttt{System.Posix.DynamicLinker} package. Instead of directly using them, I
have developed a package called \texttt{dynamic-linker-template} that uses
Template Haskell (TH) to remove most of the boilerplate code. TH allows to
perform meta-programming and to generate Haskell code given the code already
parsed in a module. In my package, I use it to get the field names of a data
type and to generate the code that loads symbols with these names from a
library dynamically.

TODO: 
\begin{itemize}
   \item Dynamic Linking
   \item Dynamic Linker Template
   \item Platform Configuration (clEnableCPU, ...)
   \item Hybrid Host/OpenCL\_CPU buffer
   \begin{itemize}
      \item Hybrid = HB HostBuffer | CLB OpenCLBuffer
      \item Method to transform one into the other (CL\_MEM\_USE\_HOST\_PTR)
      \item Should be cost-free...
      \item Intel OpenCL: bad alignment = duplication
   \end{itemize}
   \item Region, Link
   \item Synchronous data transfer / asynchronous?
   \item Method to bench link speed (no NUMA support, nor NUIOA)
\end{itemize}

\summary{
In this chapter we added support for dynamic linking with the OpenCL library and
memory management (\texttt{allocate} and \texttt{release}) into OpenCL devices.
We introduced \texttt{Link} and \texttt{Region} types, we added a method to list
available memories in a platform and we provided a method to perform data
transfers on links.

More specifically, we used:
\begin{itemize}
   \item OpenCL Platform model
   \item Template Haskell
\end{itemize}
}


\backmatter

\end{document}
