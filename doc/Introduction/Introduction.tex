%%%%%%%%%%%%%%%%%%%%%%%%%%%%%%%%%%%%%%%%%
% Large Colored Title Article
% LaTeX Template
% Version 1.1 (25/11/12)
%
% This template has been downloaded from:
% http://www.LaTeXTemplates.com
%
% Original author:
% Frits Wenneker (http://www.howtotex.com)
%
% License:
% CC BY-NC-SA 3.0 (http://creativecommons.org/licenses/by-nc-sa/3.0/)
%
%%%%%%%%%%%%%%%%%%%%%%%%%%%%%%%%%%%%%%%%%

%----------------------------------------------------------------------------------------
%	PACKAGES AND OTHER DOCUMENT CONFIGURATIONS
%----------------------------------------------------------------------------------------

\documentclass[DIV=calc, paper=a4, fontsize=11pt, twocolumn]{scrartcl}	 % A4 paper and 11pt font size

\usepackage{datetime}
\usepackage{lipsum} % Used for inserting dummy 'Lorem ipsum' text into the template
\usepackage[english]{babel} % English language/hyphenation
\usepackage[protrusion=true,expansion=true]{microtype} % Better typography
\usepackage{amsmath,amsfonts,amsthm} % Math packages
\usepackage[svgnames]{xcolor} % Enabling colors by their 'svgnames'
\usepackage[hang, small,labelfont=bf,up,textfont=it,up]{caption} % Custom captions under/above floats in tables or figures
\usepackage{booktabs} % Horizontal rules in tables
\usepackage{fix-cm}	 % Custom font sizes - used for the initial letter in the document

\usepackage{sectsty} % Enables custom section titles
\allsectionsfont{\usefont{OT1}{phv}{b}{n}} % Change the font of all section commands

\usepackage{fancyhdr} % Needed to define custom headers/footers
\pagestyle{fancy} % Enables the custom headers/footers
\usepackage{lastpage} % Used to determine the number of pages in the document (for "Page X of Total")

% Headers - all currently empty
\lhead{}
\chead{}
\rhead{}

% Footers
\lfoot{}
\cfoot{}
\rfoot{\footnotesize Page \thepage\ of \pageref{LastPage}} % "Page 1 of 2"

\renewcommand{\headrulewidth}{0.0pt} % No header rule
\renewcommand{\footrulewidth}{0.4pt} % Thin footer rule

\usepackage{lettrine} % Package to accentuate the first letter of the text
\newcommand{\initial}[1]{ % Defines the command and style for the first letter
\lettrine[lines=3,lhang=0.3,nindent=0em]{
\color{DarkGoldenrod}
{\textsf{#1}}}{}}

%----------------------------------------------------------------------------------------
%	TITLE SECTION
%----------------------------------------------------------------------------------------

\usepackage{titling} % Allows custom title configuration

\newcommand{\HorRule}{\color{DarkGoldenrod} \rule{\linewidth}{1pt}} % Defines the gold horizontal rule around the title

\pretitle{\vspace{-30pt} \begin{flushleft} \HorRule \fontsize{50}{50} \usefont{OT1}{phv}{b}{n} \color{DarkRed} \selectfont} % Horizontal rule before the title

\title{ViperVM} % Your article title

\posttitle{\par\end{flushleft}\vskip 0.5em} % Whitespace under the title

\preauthor{\begin{flushleft}\large \lineskip 0.5em \usefont{OT1}{phv}{b}{sl} \color{DarkRed}} % Author font configuration

\author{Sylvain Henry} % Your name

\postauthor{\footnotesize \usefont{OT1}{phv}{m}{sl} \color{Black} % Configuration for the institution name
%University of California % Your institution

\par\end{flushleft}\HorRule} % Horizontal rule after the title

%\date{November 30th, 2013} % Add a date here if you would like one to appear underneath the title block
\date{\today{} - \currenttime}

%----------------------------------------------------------------------------------------

\begin{document}

\maketitle % Print the title

\thispagestyle{fancy} % Enabling the custom headers/footers for the first page 

%----------------------------------------------------------------------------------------
%	ABSTRACT
%----------------------------------------------------------------------------------------

% The first character should be within \initial{}
\initial{H}\textbf{igh-performance computing (HPC) software is notoriously hard to
write. With ViperVM we aim to provide a comprehensive HPC software stack
(compilers, runtime system, etc.) to simplify the use of high-performance
architectures (multi-core, GPUs, clusters, etc.).
}

%----------------------------------------------------------------------------------------
%	ARTICLE CONTENTS
%----------------------------------------------------------------------------------------

\section{Introduction}

Let's imagine current high-performance computer architectures have just been
invented and the only known way to program them is to use assembly languages
specific to each of them: would we invent C/C++, PThread, OpenMP, MPI, CUDA,
OpenCL? Obviously not, or at least not exactly in the same way. The purpose of
our project is to experiment with programming models and languages in the
context of high-performance computing without any prerequisite (integration into
existing applications, support for standard models, etc.). ViperVM aims to be a
fully integrated software with as few dependencies as possible that defines how
user input is to be expressed and fully controls the use of the computing
resources.


%TODO: 
%\begin{itemize}
%   \item HPC complex, many architectures and many frameworks/models/languages
%   (C/Fortran, SIMD, OpenMP, MPI, CUDA/OpenCL, Xeon Phi, CELL BE, etc.)
%   \item Higher-level approaches
%   \begin{itemize}
%      \item Actor model: Charm++, Erlang, Akka
%      \item Data-flow: StreamIt, BrookGPU
%      \item Task graph: StarPU, StarSS, Parsec
%      \item Partitioned Global Address Space (PGAS): X10, Chapel/ZPL
%      \item Data-parallel: Accelerate, Accelerator, ArBB
%      \item Parallel Functional Programming: Eden
%      \item Logic: Prolog
%   \end{itemize}
%   \item Aims of our project
%   \begin{itemize}
%      \item Common framework for every model
%      \item Easy to use and to extend
%      \item High-performance on every architecture
%      \item Free software
%   \end{itemize}
%\end{itemize}


%------------------------------------------------

\section{Overview}




\section{Themes}

\subsection{Platform discovery}

\begin{itemize}
   \item Retrieve hardware capabilities and layout (memories, cores, networks).
   \item Provide an abstract representation
   \begin{itemize}
      \item Provide abstract methods (memory allocation/release, communications, etc.)
   \end{itemize}
\end{itemize}

\subsection{Computational kernels}

\begin{itemize}
   \item Computational kernel: code that only performs computations (no
   syscall\ldots)
   \item Provide efficient computational kernels for each architecture
   \item Generation from high-level representations
   \item Cost models
   \item Auto-tuning and benchmarking
\end{itemize}

\subsection{High-level programming models}

\begin{itemize}
   \item Task graph
   \item Parallel functional programming
\end{itemize}

\subsection{Kernel scheduling}

\begin{itemize}
   \item Cost models and heuristics
   \item Prefetching strategies
\end{itemize}

\subsection{Fault-tolerance}

\begin{itemize}
   \item Check-pointing
\end{itemize}

\subsection{Debug and profiling}

\begin{itemize}
   \item Traces
\end{itemize}

\end{document}
