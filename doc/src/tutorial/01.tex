\chapter{Architecture Model: Memory}

The first thing we need to decide is how the architecture will be represented
into the environment. 

In the OpenCL specification, an architecture is composed of several platforms,
themselves composed of several devices. Each device contains one memory and can
execute kernels. Obviously, it is flawed: platforms make no sense and are not
interoperable, device can contain several memories and processing units of
different kinds, etc. We will try to do a better job at defining an architecture
model.

The first entity we define is \texttt{Memory}. Perhaps a better would be
\texttt{Storage}. Each memory should be identifiable so we give them a unique
identifier. In addition, as there are many kinds of memory units
involved in architectures (hard disks, RAM, OpenCL device memory, etc.), we
introduce the algebraic data type \texttt{MemoryPeer} to store driver
specific details. We will add parameters to the constructor later for each
driver.

\begin{lstlisting}
type ID = Int

data Memory = Memory {
   memoryId :: ID,
   memoryPeer :: MemoryPeer
}

data MemoryPeer =
     HostMemory
   | CUDAMemory
   | OpenCLMemory
   | DiskMemory
\end{lstlisting}

Basically, the only thing a program can do with memory is to associate a region
of it with an opaque identifier: this is called allocating and the result is
called a buffer. Removing this association is called (buffer) releasing. Hence
we define a Buffer entity. Depending on the kind of memory we consider, the
low-level identifier will be different. For instance, for the host memory and
CUDA we get a pointer, for OpenCL we get an opaque clMem entity, for a hard disk
we get a file name\ldots These different identifiers are represented with the
algebraic data type \texttt{BufferPeer}.

\begin{lstlisting}
import Foreign.Ptr

data Buffer = Buffer {
   bufferMemory :: Memory,
   bufferPeer :: BufferPeer
}

data BufferPeer = 
     HostBuffer (Ptr ())
   | CUDABuffer
   | OpenCLBuffer
   | DiskBuffer
\end{lstlisting}

We would like to keep track of buffers that have been allocated in each memory.
To do that, we add a mutable field to the Memory type. We want to use software
transactional memory (STM) to manage future concurrent accesses to it, so we use
a \texttt{TVar} type. Don't forget to add the \texttt{stm} package to the list
of dependencies.

\begin{lstlisting}
import Control.Concurrent.STM

data Memory = Memory {
   ...
   memoryBuffers :: TVar [Buffer]
}
\end{lstlisting}

Now we can provide our first memory primitives: allocate and release. The code
for each driver is different, but for now we only code it for the host memory.
We use \texttt{undefined} to fill the blank and have a program that compiles
anyway. An allocation is only parameterized by the size of the desired memory
region and by the memory into which the allocation is performed. It returns
either the allocated buffer or an error. We decide to store the allocation size
in the buffer, hence the additional field. Finally, we set the type of a buffer
size to be \texttt{Word64} so that it should be large enough on most
architectures.

\begin{lstlisting}
import Data.Word

data Buffer = Buffer {
   ...
   bufferSize :: BufferSize
   ...
}

type BufferSize = Word64

data Result a b = Success a | Error b

data AllocError = 
     ErrAllocOutOfMemory
   | ErrAllocUnknown

allocate :: Word64 -> Memory -> IO (Result Buffer AllocError)
allocate size mem = do

   bufPeer <- case (memoryPeer mem) of
      HostMemory   -> allocateHost size mem
      CUDAMemory   -> undefined
      OpenCLMemory -> undefined
      DiskMemory   -> undefined

   case bufPeer of
      Error err -> return (Error err)
      Success peer -> do
         let buf = Buffer mem size peer
         atomically $ do
            bufs <- readTVar (memoryBuffers mem)
            writeTVar (memoryBuffers mem) (buf:bufs)
         return (Success buf)
\end{lstlisting}

Finally the "host" specific code to allocate a buffer is:

\begin{lstlisting}
{-# LANGUAGE ForeignFunctionInterface #-}

import Foreign.C.Types

foreign import ccall unsafe "stdlib.h malloc"  malloc :: CSize -> IO (Ptr a)
foreign import ccall unsafe "stdlib.h free"    free   :: Ptr a -> IO ()

-- | Allocate a buffer in host memory
allocateHost :: Word64 -> Memory -> IO (Result BufferPeer AllocError)
allocateHost size _ = do
   ptr <- malloc (fromIntegral size)
   return $ if ptr == nullPtr
      then Error ErrAllocOutOfMemory
      else Success (HostBuffer ptr)
\end{lstlisting}


To simplify the code a little bit, we introduce the \texttt{onSuccess} function
and we replace the second \texttt{case} in \texttt{allocate} with a call to it.

\begin{lstlisting}
import Control.Applicative ( (<$>) )

onSuccess :: Result a b -> (a -> IO c) -> IO (Result c b)
onSuccess (Success a) f = Success <$> f a
onSuccess (Error b) _ = return (Error b)

allocate size mem = do
...
   onSuccess bufPeer $ \peer -> do
      let buf = Buffer mem size peer
      atomically $ do
         bufs <- readTVar (memoryBuffers mem)
         writeTVar (memoryBuffers mem) (buf:bufs)
      return buf
\end{lstlisting}

Finally we can add the \texttt{release} primitive. Before we need to remove the
buffer from the list of allocated buffers of the memory it is allocated in.
Hence, we need to be able to find the buffer in the list and to delete it. To do
that, \texttt{Buffer} must be an instance of the \texttt{Eq} type class. We
first create an instance of \texttt{Eq} for \texttt{Memory} as it is easy given
there is a unique identifier field. Then we can automatically derive the
\texttt{Eq} instances for \texttt{BufferPeer} and \texttt{Buffer}.

\begin{lstlisting}
import Data.List

instance Eq Memory where
   (==) a b = memoryId a == memoryId b

data Buffer = Buffer {
   ...
} deriving (Eq)

data BufferPeer = ... deriving (Eq)

release :: Buffer -> IO ()
release buf = do
   atomically $ do
      let bufsVar = memoryBuffers (bufferMemory buf)
      bufs <- readTVar bufsVar
      writeTVar bufsVar (delete buf bufs)

   case bufferPeer buf of
      HostBuffer ptr -> free ptr
      CUDABuffer     -> undefined
      OpenCLBuffer   -> undefined
      DiskBuffer     -> undefined
\end{lstlisting}

Here is the full listing of our current \texttt{Platform} module.

\begin{lstlisting}
{-# LANGUAGE ForeignFunctionInterface #-}
module ViperVM.Platform.Platform (
   Memory(..), Buffer(..), Result(..),
   MemoryPeer(..), BufferPeer(..), AllocError(..),
   allocate, release
) where

import Control.Applicative ( (<$>) )
import Control.Concurrent.STM
import Data.List
import Data.Word
import Foreign.Ptr
import Foreign.C.Types

type ID = Int

data Memory = Memory {
   memoryId :: ID,
   memoryPeer :: MemoryPeer,
   memoryBuffers :: TVar [Buffer]
}

instance Eq Memory where
   (==) a b = memoryId a == memoryId b

data MemoryPeer =
     HostMemory
   | CUDAMemory
   | OpenCLMemory
   | DiskMemory

data Buffer = Buffer {
   bufferMemory :: Memory,
   bufferSize :: BufferSize,
   bufferPeer :: BufferPeer
} deriving (Eq)

data BufferPeer = 
     HostBuffer (Ptr ())
   | CUDABuffer
   | OpenCLBuffer
   | DiskBuffer
   deriving (Eq)

type BufferSize = Word64

data Result a b = Success a | Error b

onSuccess :: Result a b -> (a -> IO c) -> IO (Result c b)
onSuccess (Success a) f = Success <$> f a
onSuccess (Error b) _ = return (Error b)

data AllocError = 
     ErrAllocOutOfMemory
   | ErrAllocUnknown

-- | Allocate a buffer of the given size in the memory 
allocate :: Word64 -> Memory -> IO (Result Buffer AllocError)
allocate size mem = do

   bufPeer <- case (memoryPeer mem) of
      HostMemory   -> allocateHost size mem
      CUDAMemory   -> undefined
      OpenCLMemory -> undefined
      DiskMemory   -> undefined

   onSuccess bufPeer $ \peer -> do
      let buf = Buffer mem size peer
      atomically $ do
         bufs <- readTVar (memoryBuffers mem)
         writeTVar (memoryBuffers mem) (buf:bufs)
      return buf

foreign import ccall unsafe "stdlib.h malloc"  malloc :: CSize -> IO (Ptr a)
foreign import ccall unsafe "stdlib.h free"    free   :: Ptr a -> IO ()

-- | Allocate a buffer in host memory
allocateHost :: Word64 -> Memory -> IO (Result BufferPeer AllocError)
allocateHost size _ = do
   ptr <- malloc (fromIntegral size)
   return $ if ptr == nullPtr
      then Error ErrAllocOutOfMemory
      else Success (HostBuffer ptr)

-- | Release a buffer
release :: Buffer -> IO ()
release buf = do
   atomically $ do
      let bufsVar = memoryBuffers (bufferMemory buf)
      bufs <- readTVar bufsVar
      writeTVar bufsVar (delete buf bufs)

   case bufferPeer buf of
      HostBuffer ptr -> free ptr
      CUDABuffer     -> undefined
      OpenCLBuffer   -> undefined
      DiskBuffer     -> undefined
\end{lstlisting}

We can test that it is quite working with the following GHCI session:
\begin{lstlisting}
> import ViperVM.Platform.Platform
> import Control.Concurrent.STM
> import Control.Applicative
> m <- Memory 1 HostMemory <$> atomically (newTVar [])
> Success b <- allocate 10000 m
> length <$> atomically (readTVar (memoryBuffers m))
1
> Error err <- allocate 100000000000000 m
> length <$> atomically (readTVar (memoryBuffers m))
1
> release b
> length <$> atomically (readTVar (memoryBuffers m))
0
\end{lstlisting}
