\documentclass[twocolumn]{article}

\usepackage[xcolor]{rvdtx}

%\usepackage[fontsize=8pt,baseline=9.6pt,lines=50]{grid}
%\usepackage[fontsize=9pt,baseline=10.8pt]{grid}
\usepackage[fontsize=10pt,baseline=12pt,lines=53]{grid}
%\usepackage[fontsize=11pt,baseline=13.2pt]{grid}
%\usepackage[fontsize=12pt,baseline=14.4pt]{grid}
%\usepackage[fontsize=20pt,baseline=24pt,lines=20]{grid}

\newcommand{\ip}[2]{(#1, #2)}
\columnsep=20pt
\begin{document}

\title{ViperVM -- Memory and Data Management}
\author{Sylvain HENRY}
\contact{hsyl20@gmail.com}
\version{1.0}
\date{2014/10/31}
%\keywords{\LaTeX, grid typesetting}

\newtheorem{defin}{Definition}

\maketitle

\section{Introduction}

ViperVM automatically manages heterogeneous distributed memory.

\section{Layers}

\subsection{Allocation}

On systems with paging, memory has to be allocated by the application to be
used. An allocation basically reserves a set of pages that are or will later be
mapped on physical memory. We call the result of an allocation a \emph{buffer}.

\begin{defin}
A \emph{buffer} is a contiguous indexed set of cells characterized by:
\begin{itemize}
   \item its size
   \item the memory it is allocated in
\end{itemize}
\end{defin}

On systems without paging, all of the memory is directly accessible. A single
buffer could be used to encompass the whole memory.

\subsection{Data}

Buffers can contain arbitrary data in their cells. However, we want the runtime
system to have a representation of the data contained into them. \emph{Data} are
used to describe some cells of a buffer.

\begin{defin}
A \emph{data} is characterized by
\begin{itemize}
   \item the buffer it is in
   \item the index of the first cell of the buffer described by the data
   \item a data layout
\end{itemize}
\end{defin}

\begin{defin}
A \emph{data layout} consists in the description of several cells. It can be:
\begin{itemize}
   \item A scalar: Float, Double, Signed/Unsigned Integer of arbitrary precision
   (in bytes). Parameterize with the endianness.
   \item An array with a fixed length: parameterized with a data layout used for
   each cell
   \item A structure: a vector of data layouts
   \item A number of padding bytes: valid or invalid bytes that must be skipped
   in arrays of structures
\end{itemize}
\end{defin}

Scalars are parameterized with their endianness (the order of the bytes
composing them). This is because data can be transferred from on memory to
another which does not use the same kind of byte ordering. Without explicitly
performing the byte reordering, scalars cannot be used directly in the new
memory except for single byte ones. With data layouts, the runtime system has
enough information to automatically perform this reordering or to generate
kernels to do it.

The distinction between valid and invalid padding bytes in a layout can be used
by the runtime system for instance to optimize data transfers: it may be cheaper
to perform a single transfer overwriting invalid padding bytes than to avoid
overwriting them (e.g. strided transfer or several transfers).

\begin{defin}
A \emph{region} is a untyped set of cells in a buffer. We currently support two
kinds of region:
\begin{itemize}
   \item 1D region: contiguous set of cells characterized by an offset and a
   number of cells.
   \item 2D region: strided set of cells characterized by an offset, a row size,
   a number of rows and a number of padding cells between two rows.
\end{itemize}
\end{defin}

Regions are used to perform data transfers. A data can easily be converted into
a region by using its layout. DMA (direct-memory access) modules may support
transfers with strides which correspond to 2D regions. It may not be required
that the source and target numbers of padding cells are equal.

\subsection{Object}

An object is an \emph{immutable} abstract data representation with its own
parameters. It can have several concrete instances which are Data parametrized
with representation parameters.

For instance, a Matrix object is parameterized by its cell type and
the number and sizes of its dimensions. It is an abstract data because
conceptually it is just a function from the index domain (given by its
dimensions) to the codomain (given by its cell type). However its concrete
instances can be: 
\begin{itemize}
   \item A dense array: parameters of the representation define
dimension storage order (row major, column major, generalized to any number of
dimensions)
   \item A sparse representation: compressed row-storage, etc.
\end{itemize}

An object can also be created from other source objects in an abstract way. For
instance, a sub-matrix can be defined by a source matrix and (offset,range)
couples for each dimensions.

\begin{defin}
An object is characterized by
\begin{itemize}
   \item The type of its parameters
   \item A list of concrete data instances each associated to representation parameters
   \item A list of object sources each associated with source parameters
   \item A list of object targets (objects using this one as a source)
\end{itemize}
\end{defin}


\section{Garbage Collection}

\subsection{Data}

Data cannot contain references (pointers) to other data. Hence there cannot be
any cycle and a kind of reference counting algorithm can be used. They can only
be referenced by one object at a time, so they live at most as long as their
associated object.

A data can be detached from its object iff the object has at least one other
concrete data instance or has at least one object source. Detaching a data can
be used to release it or to perform a transformation and before attaching it to
another object.

\subsection{Objects}

Objects can reference other objects (as sources or targets). The link (A,B) is:
\begin{itemize}
   \item strong: object A has no concrete instance and has only object B as
   source
   \item weak: object A has object B as source but also have at least one concrete
   instances and/or other object sources
\end{itemize}

If for all $x$, links ($x$, B) are weak, B can be released and links must be
removed. Otherwise, before releasing B, all the $x$ objects so that the link
($x$, B) is weak must create a concrete instance from B.

\end{document}  

% End of document.
